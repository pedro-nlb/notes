\documentclass[11pt,A4]{article}
\usepackage[T1]{fontenc}
\usepackage[utf8]{inputenc}
\usepackage[UKenglish]{babel}
\usepackage{mathtools}
\usepackage{amsthm}
\usepackage{amssymb}
%\usepackage{mathrsfs}
\usepackage[mathscr]{euscript}
\usepackage{enumitem}
\usepackage{tikz-cd}
\usepackage{hyperref}
\usepackage[noabbrev]{cleveref}
\usepackage{todonotes}
\usepackage{bbm}

% Plain theorems
\theoremstyle{plain}
\newtheorem{thm}{Theorem}[section]
\newtheorem{lm}[thm]{Lemma}
\newtheorem{prop}[thm]{Proposition}
\newtheorem{cor}[thm]{Corollary}

% Definition theorems
\theoremstyle{definition}
\newtheorem{defn}[thm]{Definition}
\newtheorem{exa}[thm]{Example}

% Remark theorems
\theoremstyle{remark}
\newtheorem{rem}[thm]{Remark}
\newtheorem{q}[thm]{Question}

% Mathbb
\newcommand{\N}{\mathbb{N}}
\newcommand{\Z}{\mathbb{Z}}
\newcommand{\1}{\mathbbm{1}}

% Mathscr for categories
\newcommand{\C}{\mathscr{C}}
\newcommand{\Top}{\mathscr{T}op}
\newcommand{\CHaus}{\mathscr{CH}aus}
\newcommand{\CG}{\mathscr{CG}}
\newcommand{\Ab}{\mathscr{A}b}
\newcommand{\Set}{\mathscr{S}et}
\newcommand{\D}{\mathscr{D}}
\newcommand{\Db}{\mathscr{D}^{\mathrm{b}}}
\newcommand{\QCoh}{\mathscr{QC}oh}
\newcommand{\Coh}{\mathscr{C}oh}

% Math operators
\DeclareMathOperator{\Hom}{Hom}
\DeclareMathOperator{\Cond}{Cond}
\DeclareMathOperator{\Ob}{Ob}
\DeclareMathOperator{\im}{im}
\DeclareMathOperator{\PSh}{PSh}
\DeclareMathOperator{\Sh}{Sh}

% New commands
\newcommand{\pe}{*_{pro\acute et}}
\renewcommand{\u}[1]{\underline{#1}}
\newcommand{\ot}{\otimes}
\newcommand{\op}{\oplus}
\newcommand{\fp}[1]{\times_{#1}}
\newcommand{\id}{\mathrm{id}}
\newcommand{\ev}{\mathrm{ev}}


\title{Various lecture notes}
\author{Pedro Núñez}
\date{\today}

\begin{document}

\maketitle

\tableofcontents

\section{[CM] Talk 1 (Johan Commelin): Condensed Sets - 21.10.19}

Motivation: topological abelian groups do not form an abelian category.

\begin{exa}
    $\mathbb{R}_{disc}\to \mathbb{R}$ is epi and mono, but not iso.
\end{exa}

Another motivation is coherent duality:

\begin{thm}
    Let $f\colon X\to Y$ be a proper or quasi-projective morphism of Noetherian schemes of finite Krull dimension. Then there exists a right adjoint $f^{!}$ to the derived direct image functor $f_{!}=Rf_{*}\colon \Db(\QCoh(X))\to \Db(\QCoh(Y))$.
\end{thm}

At some point analytic rings will come up.
We will then look at the category of solid modules, in which the 6-functor formalism works nicer than in the classical setting (e.g. when $f_{!}$ is not defined in the classical setting, $f_{!}$ takes non-discrete values in the condensed settings, which are "not there" in the classical setting).

\begin{defn}
    Pro\'{e}tale site of a point, denoted $\pe$, is the category of profinite sets with finite jointly surjective families of continuous maps as covers.
    A \textit{condensed set} (resp. group, ring, ...) is a sheaf of sets (resp. groups, rings, ...) on $\pe$.
    We denote by $\Cond(\C)$ the category of condensed objects of a category $\C$.
\end{defn}

\begin{defn}
    A \textit{condensed set} (resp. group, ring, ...) is a contravariant functor $X$ from $\pe$ to the category of sets (resp. groups, rings, ...) such that 
    \begin{enumerate}[label=\roman*)]
	\item $X(\varnothing)=*$.
	\item For all profinite sets $S_{1}$ and $S_{2}$ the natural map
	    \[ X(S_{1}\sqcup S_{2})\to X(S_{1})\times X(S_{2}) \]
	    is an isomorphism.
	\item For any surjection of profinite sets $f\colon S'\twoheadrightarrow S$ we get an induced\footnote{Since the pullback diagram is commutative, the image of $X(f)$ is indeed induces a morphism as claimed.} isomorphism
	    \[ X(S)\to \{ x\in X(S')\mid \pi_{1}^{*}(x)=\pi_{2}^{*}(x)\in X(S'\fp{S}S')\} \]
    \end{enumerate}
\end{defn}

We will call $X(*)$ the \textit{underlying object} in $\C$ of a condensed object.

\begin{rem}
    We will use $T$ for topological spaces vs. $X,Y$ for condensed sets, as opposed to Scholze's mixing of those notations.
\end{rem}

\subsection{Recollections on sheaves on sites}

Let $F$ be a presheaf on a site, which is just a contravariant functor to whatever category in which our sheaves are gonna take values.
If $U=\cup_{i}U_{i}$ is an open cover, the topological sheaf axiom could be phrased as: $F(U)$ is an equalizer of the diagram
\[ \prod_{i}F(u_{i})\rightrightarrows \prod_{i,j}F(U_{i}\cap U_{j}). \]
Note that $U_{i}\cap U_{j}$ is just the fiber product of the two inclusions.

\begin{defn}[Coverage]
    See definition 2.1 in \href{https://ncatlab.org/nlab/show/coverage}{nCat}.
\end{defn}

\begin{defn}
    $F$ a presheaf on $\C$.
    A collection $(s_{i})\in \prod_{i}F(U_{i})$ for $\{f_{i}\colon U_{i}\to U\}$ a covering is called a \textit{matching family} if for all $h\colon V\to U$ we have $g^{*}(s_{i})=h^{*}(s_{j})$ for $g$ and $h$ in the diagram
    \begin{center}
	\begin{tikzcd}
	    V\arrow{r}{h}\arrow{d}{g} & U_{j}\arrow{d}{f_{j}} \\
	    U_{i}\arrow{r}{f_{i}} & U
	\end{tikzcd}
    \end{center}
\end{defn}

\begin{defn}
    $F$ is a sheaf with respect to $\{U_{i}\to U\}$ if for all matching families $(s_{i})$ there exists a unique $s\in F(U)$ such that $f_{i}^{*}(s)=s_{i}$.
    We say that $F$ is a \textit{sheaf} if it is a sheaf for all covering families.
\end{defn}

\begin{rem}
    A sheaf of abelian groups is just a commutative group object in the category of sheaves of sets.
\end{rem}

\begin{thm}
    If $\C$ is a site, then $\Ab(\C)$ is an abelian category.
\end{thm}

\begin{defn}
    An additive category is a category in which the hom-sets are endowed with an abelian group structure in a way that makes composition bilinear and such that finite biproducts exist.
\end{defn}

Recall Grothendieck's axioms:
AB1) Every morphism has a kernel and a cokernel.
AB2) For every $f\colon A\to B$, the natural map $\operatorname{coim}(f)\to \im{f}$ is an iso.
AB3) All colimit exist.
AB4) AB3) + arbitrary direct sums are exact.
AB5) AB3) + arbitrary filtered colimits are exact.
AB6) AB3) + $J$ an index set, $\forall j\in J$ a filtered category (think of directed set) $I_{j}$, functors $M\colon I_{j}\to \C$, then
\[\varinjlim_{(i_{j}\in I_{j})_{j}}\prod_{j} M_{i_{j}}\to \prod_{j\in J}\varinjlim_{i_{j}\in I_{j}} M_{i_{j}} \]

\begin{thm}
    $\C$ a site. Then $\Ab(\C)$ satisfies AB3), AB4), AB5) and AB6).
\end{thm}

In fact, our case is even nicer:

\begin{thm}
    $\Cond(\Ab)$ in addition satisfies AB6) and AB4*).
\end{thm}

\subsection{Compactly generated topological spaces}

\begin{defn}
    A topological space $T$ is called \textit{compactly generated} if any function $f\colon T\to T'$ is continuous as soon as the composite $S\to T\to T'$ is continuous for all maps $S\to T$ where $S$ is compact and Hausdorff.
    See also \href{https://ncatlab.org/nlab/show/compactly+generated+topological+space}{nCat}.
\end{defn}

The inclusion functor $\CG \hookrightarrow \Top$ has a right adjoint $(-)^{cg}$.
If $T$ is any topological space, then the topology on $T^{cg}$ is the finest topology on $T$ such that $\sqcup_{S\to T}S\to T$ is continuous, where $S$ ranges over all compact Hausdorff spaces.

Let $T$ be a topological space.
We view $T$ as a presheaf on $\pe$ by setting $T(S)=\Hom_{\Top}(S,T)$ for all profinite sets $S$.
We denote this by $\u{T}$.
Claim: $\u{T}$ is a sheaf.
\begin{enumerate}[label=\roman*)]
    \item The first condition $\u{T}(\varnothing)=*$ is true, because there is exactly one morphism from the empty set to any topological space.
    \item $\u{T}(S_{1}\sqcup S_{2})=\u{T}(S_{1})\times \u{T}(S_{2})$ by universal property of disjoint union.
    \item For any surjection $S'\twoheadrightarrow S$ we get an isomorphism
	\[ \u{T}(S)\to \{ x\in \u{T}(S')\mid \pi_{1}^{*}(x)=\pi_{2}^{*}(x)\in \u{T}(S'\fp{S}S')\}\]
\end{enumerate}

Since $\Top\to \Cond(\Set)$ preserves products, group objects are preserved, so it maps topological groups to condensed groups etc.

\begin{prop}
    \begin{enumerate}[label=\roman*)]
	\item This functor is faithful and fully faithful when restricted to the full subcategory of compactly generated spaces.
	\item It admits a left adjoint $X\mapsto X(*)_{top}$ where $X(*)_{top}$ gets the quotient topology of $\sqcup_{S\to X}S\to X(*)$ as above.
	    The counit $I(*)_{top}\to T$ agrees with $T^{cg}\to T$.
    \end{enumerate}
\end{prop}

Coming back to our original example:

\begin{exa}
    $\mathbb{R}_{disc}\to \mathbb{R}$ can be seen in the condensed world as $\u{\mathbb{R}_{disc}}\to \u{\mathbb{R}}$, i.e. from locally constant functions to continuous functions.
    This is still a mono, but now it is not an epi.
    The cokernel $Q$ can be described as $Q(S)=\{ S\to \mathbb{R}\text{ continuous }\}/\{ S\to \mathbb{R}\text{ locally constant }\}$.
    Note in particular that the underlying set of $Q$ is just $*$, reflecting the fact that the cokernel was trivial in the classical setting.
\end{exa}

\section{[LT] Lecture 1 - 22.10.19}

Today: big picture.

An \textit{algebraic variety} is the solution set of a family of polynomial equations in $\mathbb{C}^{n}$.
For example, if $f(x,y,z,t)=xy-tz$, then
\[ \mathbb{V}(f)=\{(x,y,z,t)\in \mathbb{C}^{4}\mid xy-tz=0\} \]
is an algebraic variety in $\mathbb{C}^{4}$.
Another example would be the parabola $\{ y-x^{2}=0\}\subseteq \mathbb{C}^{2}$.

We can think of $\mathbb{V}(f)$ as a family of varieties parametrized by the variable $t$.
For $t=1$ we get the equation $xy-z=0$ in $\mathbb{C}^{3}$.
We can perform a change of coordinates $(x,y)\mapsto (x+iy,x-iy)$ to turn our equation into $x^{2}+y^{2}=z$.
For $z=0$, the variety $X_{0}=\{x^{2}+y^{2}=0\}$ has an \textit{ordinary double point} at the origin [picture: cone] (a.k.a. node if we think of $X_{0}$ as a curve\footnote{These are $1$-dimensional complex varieties, so topologically they are surfaces.}.
These are a particularly nice kind of singularities\footnote{Singularities will appear naturally while studying the topology of algebraic varieties.}.
For $z\neq 0$ we get the equation $xy-1=0$.
This is a ruled surface $X_{z}$ [picture: chimeny of nuclear plant with a loop $\gamma$ at its base].
As $z\mapsto 0$, the central loop $\gamma$ contract to the ordinary double point.
We have a projection $\pi\colon \mathbb{V}(f)\to \mathbb{C}$, and Ehresmann's lemma tells us that for all disk $D\subseteq \mathbb{C}$ not containing $0$ we have $\pi^{-1}(D)\cong D\times X_{z_{0}}$ for any $z_{0}\in D$.

Global picture: given an arbitrary nonsingular alg. variety $X\subseteq \mathbb{C}^{n}$, can we find a map $\pi\colon X\to \mathbb{C}$ such that the fibres $X_{t}$ are nonsingular for all but finitely many $t\in \mathbb{C}$ and such that the singular fibres have at worst ODP singularities?

Problem: we are missing information "at infinity", e.g. $y=x^{2}$ versus $xy=1$.
The solution is to replace $\mathbb{C}^{n}$ by $\mathbb{C}\mathbb{P}^{n}$.

Let $X\subseteq\mathbb{P}^{n}$ be a nonsingular projective variety.

\begin{thm}
    There exists a family $(H_{t})_{t\in \mathbb{C}\mathbb{P}^{1}}$ of hyperplanes in $\mathbb{C}\mathbb{P}^{n}$ such that
    \begin{enumerate}
	\item $X\subseteq \cup_{t}H_{t}$.
	\item $X_{t}=X\cap H_{t}$ is nonsingular except for finitely many "critical values" of $t$.
	\item $X_{t}$ has ODP singularities for each critical value $t$.
    \end{enumerate}
\end{thm}

We call $(X_{t})_{t}$ a \textit{Lefschetz pencil}.
We get a rational map $X\mapsto \mathbb{C}\mathbb{P}^{1}$ sending $x\mapsto t$ whenever $x\in X_{t}$.
This is not well-defined at $x\in \cap_{t}X_{t}$, but we can arrange for $(X_{t})_{t}$ so that $\cap_{t}X_{t}=X_{0}\cap X_{\infty}$.
Blowing-up a suitable subvariety of $X$ we get maps $\tilde{X}\xrightarrow{\pi} \mathbb{C}\mathbb{P}^{1}$ and $\tilde{X}\to X$ as we wanted.
[picture A]

Some applications:

\begin{thm}[Lefschetz Hyperplane theorem]
    $X\subseteq Y\subseteq \mathbb{C}\mathbb{P}^{N}$ nonsingular varieties with $X$ a hypersurface in the $n$-dimensional variety $Y$, then
    \[ H_{*}(X)\to H_{*}(Y) \]
    is an isomorphism for $*<n-1$ and a surjection for $*=n-1$.
\end{thm}

In particular, if $Y=\mathbb{C}\mathbb{P}^{n}$, we have
\[ H_{*}(\mathbb{C}\mathbb{P}^{n})=\Z \text{ if $*$ is even or } 0 \text{ otherwise}.\]
If $X\subseteq \mathbb{C}\mathbb{P}^{n}$ is a nonsingular hypersurface, then its homology will be that of projective sapce on all digrees other than $n-1$.
Its $n-1$ homology will depend on the variety, e.g. the ODP (trivial $1$-homology) vs the ruled surface (with $\gamma$ non trivial on $1$-homology) of before.

\begin{exa}[Lefschetz pencil]
    $X$ elliptic curve in $\mathbb{C}\mathbb{P}^{2}$ given by $y^{2}=x(x-1)(x-\lambda)$ for $\lambda\neq 0$.
    $L=\mathbb{C}\mathbb{P}^{1}\subseteq \mathbb{C}\mathbb{P}^{2}$.
    $P\in \mathbb{C}\mathbb{P}^{1}\setminus (X\cup L)$.
    We get $X\xrightarrow{\pi}\mathbb{C}\mathbb{P}^{1}$ once we choose a square root of $x(x-1)(x-\lambda)$.
    [Picture B]
\end{exa}

\bibliographystyle{alpha}
\bibliography{refs}

\end{document}
